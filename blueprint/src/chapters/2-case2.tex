% The main TeX macros that relate your TeX code to your Lean code are:
%
% * `\lean` that lists the Lean declaration names corresponding to the surrounding
%   definition or statement (including namespaces).
% * `\leanok` which claims the surrounding environment is fully formalized. Here
%   an environment could be either a definition/statement or a proof.
% * `\uses` that lists LaTeX labels that are used in the surrounding environment.
%   This information is used to create the dependency graph. Here
%   an environment could be either a definition/statement or a proof, depending on
%   whether the referenced labels are necessary to state the definition/theorem
%   or only in the proof.
%
% The example below show those essential macros in action, assuming the existence of
% LaTeX labels `def:immersion`, `thm:open_ample` and `lem:open_ample_immersion` and
% assuming the existence of a Lean declaration `sphere_eversion`.
%
% ```latex
% \begin{theorem}[Smale 1958]
%   \label{thm:sphere_eversion}
%   \lean{sphere_eversion}
%   \leanok
%   \uses{def:immersion}
%   There is a homotopy of immersions of $𝕊^2$ into $ℝ^3$ from the inclusion map to
%   the antipodal map $a : q ↦ -q$.
% \end{theorem}
%
% \begin{proof}
%   \leanok
%   \uses{thm:open_ample, lem:open_ample_immersion}
%   This obviously follows from what we did so far.
% \end
% ```

\chapter*{Case 2}
%\label{chap:case2}
\addcontentsline{toc}{chapter}{Case 2}

\begin{lemma}
    \label{lmm:three_dvd_gcd_of_dvd_a_of_dvd_b}
    \lean{three_dvd_gcd_of_dvd_a_of_dvd_b}
    \leanok
    %\uses{}
    Let $a, b, c \in \N$. \\
    If $3 \divides a$ and $3 \divides b$ and $a ^ 3 + b ^ 3 = c ^ 3$, then $3 \divides \gcd(\set{a, b, c\})$.
\end{lemma}
\begin{proof}
    \leanok
\end{proof}

\begin{lemma}
    \label{lmm:three_dvd_gcd_of_dvd_a_of_dvd_c}
    \lean{three_dvd_gcd_of_dvd_a_of_dvd_c}
    \leanok
    %\uses{}
    Let $a, b, c \in \N$. \\
    If $3 \divides a$ and $3 \divides c$ and $a ^ 3 + b ^ 3 = c ^ 3$, then $3 \divides \gcd(\set{a, b, c\})$.
\end{lemma}
\begin{proof}
    \leanok
\end{proof}

\begin{lemma}
    \label{lmm:three_dvd_gcd_of_dvd_b_of_dvd_c}
    \lean{three_dvd_gcd_of_dvd_b_of_dvd_c}
    \leanok
    %\uses{}
    Let $a, b, c \in \N$. \\
    If $3 \divides b$ and $3 \divides c$ and $a ^ 3 + b ^ 3 = c ^ 3$, then $3 \divides \gcd(\set{a, b, c\})$.
\end{lemma}
\begin{proof}
    \leanok
\end{proof}