\chapter{Introduction}
%\label{chap:intro}
%\addcontentsline{toc}{chapter}{Introduction}

% TO SUBMIT
% – Authors: Pietro Monticone
% – References: all technical and popular articles and books
% – Citations: everything (FLT3-Regular, FLT, Mathlib, MIL, formalization in general, etc.)
% – Introduction: all relevant links and references, main mathematical results, main mathematical methods, main structures, notation, synthesis on formalisation with main benefits, etc.
% – Motivation: enrich Mathlib, dependency of / used by FLT by Buzzard et al. (Al technical and popular articles)
% – My contributions: about 20 formalized proofs, all informalised proofs, all blueprint, PR to Mathlib (), Porting to Mathlib, see LaTeX comments
% – Acknowledgements: supervision by Riccardo Brasca (let him see it before submitting), collaboration with my team (see blog post by David)

% TO PUBLISH
% – Authors: all
% – References: all technical and popular articles and books
% – Citations: only mathematical and Mathlib-relevant
% – Introduction: all relevant links and references, main mathematical results, main mathematical methods, main structures, notation
% – Motivation: enrich Mathlib, dependency of / used by FLT by Buzzard et al. (Al technical and popular articles)
% – My contributions: NO
% – Acknowledgements: supervision by Riccardo Brasca (let him see it before submitting), collaboration with my team (see blog post by David)

% MOTIVATION
% It's going to be used in the general FLT project (BUZZARD'S)

% NOTATION

% MAIN DEFINITIONS AND STRUCTURES?

% GOAL / AIM

% CONTRIBUTIONS

% PROOF STRATEGY

% Motivation: It's going to be used in the general FLT project
% – Disclaimer / Acknowledgement with my contributions
% – Informalisation should be clear
% – Introduction: definitions + theorem statements + proof strategy
% Porting to Mathlib
% Fermat's Last Theorem (FLT) is a famous problem in number theory. It states that there are no positive integers $a$, $b$, and $c$ such that $a^n + b^n = c^n$ for any integer value of $n$ greater than $2$. The theorem was first conjectured by Pierre de Fermat in 1637 and remained unproven for over 350 years. The first successful proof was given by Andrew Wiles in 1994. The proof is long and complex, and it relies on many different areas of mathematics, including algebraic geometry and modular forms.

\begin{lemma}
    \label{lmm:fermatLastTheoremWith_of_fermatLastTheoremWith_coprime}
    \lean{fermatLastTheoremWith_of_fermatLastTheoremWith_coprime}
    \leanok
    Let $R$ be a commutative semiring, domain and normalised gcd monoid.\\% ASK EXPERTS
    Let $a, b, c \in R$. \\
    Let $n \in \N$. \\\\
    Then, to prove Fermat's Last Theorem for exponent $n$ in $R$,
    one can assume, without loss of generality, that $\gcd(a,b,c)=1$.
\end{lemma}
\begin{proof}
    \leanok
    This has already been formalised and included in \href{https://pitmonticone.github.io/FLT3/docs/FLT3/Mathlib/NumberTheory/FLT/Basic.html#fermatLastTheoremWith_of_fermatLastTheoremWith_coprime}{Mathlib}.
\end{proof}

\begin{theorem}
    \label{thm:mem}
    \lean{IsCyclotomicExtension.Rat.Three.Units.mem}
    \leanok
    Let $K = \Q(\zeta_3)$ be the third cyclotomic field. \\
    Let $\cc{O}_K = \Z[\zeta_3]$ be the ring of integers of $K$. \\
    Let $\cc{O}^\times_K$ be the group of units of $\cc{O}_K$. \\
    Let $\zeta_3 \in K$ be any primitive third root of unity. \\
    Let $\eta \in \cc{O}_K$ be the element corresponding to $\zeta_3 \in K$. \\
    Let $\lambda \in \cc{O}_K$ be such that $\lambda = \eta -1$. \\
    Let $u \in \cc{O}^\times_K$ be a unit. \\\\
    Then $u \in \set{1, -1, \eta, -\eta, \eta^2, -\eta^2}$.
\end{theorem}
\begin{proof}
    \leanok
    % TODO
\end{proof}

\begin{theorem}
    \label{thm:not_exists_int_three_dvd_sub}
    \lean{IsCyclotomicExtension.Rat.Three.Units.not_exists_int_three_dvd_sub}
    \leanok
    Let $K = \Q(\zeta_3)$ be the third cyclotomic field. \\
    Let $\cc{O}_K = \Z[\zeta_3]$ be the ring of integers of $K$. \\
    Let $\cc{O}^\times_K$ be the group of units of $\cc{O}_K$. \\
    Let $\zeta_3 \in K$ be any primitive third root of unity. \\
    Let $\eta \in \cc{O}_K$ be the element corresponding to $\zeta_3 \in K$. \\
    Let $n \in \cc{O}_K$. \\\\
    Then $3 \notdivides \eta - n$.
\end{theorem}
\begin{proof}
    \leanok
    % TODO
\end{proof}

\begin{lemma}
    \label{lmm:lambda_sq}
    \lean{IsCyclotomicExtension.Rat.Three.lambda_sq}
    \leanok
    Let $K = \Q(\zeta_3)$ be the third cyclotomic field. \\
    Let $\cc{O}_K = \Z[\zeta_3]$ be the ring of integers of $K$. \\
    Let $\cc{O}^\times_K$ be the group of units of $\cc{O}_K$. \\
    Let $\zeta_3 \in K$ be any primitive third root of unity. \\
    Let $\eta \in \cc{O}_K$ be the element corresponding to $\zeta_3 \in K$. \\
    Let $\lambda \in \cc{O}_K$ be such that $\lambda = \eta -1$. \\\\
    Then $\lambda^2 = -3 \eta$.
\end{lemma}
\begin{proof}
    \leanok
    By definition we have that $\lambda = \eta -1$, which implies that
    $$\lambda^2 = (\eta - 1)^2 = \eta^2 - 2\eta + 1.$$
    Since $\eta$ corresponds to a root of the equation $x^2 + x + 1 = 0$, then $\eta^2 = -1 - \eta$.
    Substituting back, we can conclude that
    $$\lambda^2 = (-1 - \eta) - 2\eta + 1 = -3\eta.$$
\end{proof}

\begin{theorem}
    \label{lmm:eq_one_or_neg_one_of_unit_of_congruent}
    \lean{IsCyclotomicExtension.Rat.Three.eq_one_or_neg_one_of_unit_of_congruent}
    \leanok
    Let $K = \Q(\zeta_3)$ be the third cyclotomic field. \\
    Let $\cc{O}_K = \Z[\zeta_3]$ be the ring of integers of $K$. \\
    Let $\cc{O}^\times_K$ be the group of units of $\cc{O}_K$. \\
    Let $\zeta_3 \in K$ be any primitive third root of unity. \\
    Let $\eta \in \cc{O}_K$ be the element corresponding to $\zeta_3 \in K$. \\
    Let $\lambda \in \cc{O}_K$ be such that $\lambda = \eta -1$. \\
    Let $u \in \cc{O}^\times_K$ be a unit. \\\\
    If $\exists n \in \Z$ such that $\lambda^2 \divides u - n$, then
    $u = 1 \lor u = -1$. \\
    This is a special case of the Kummer's Lemma.
\end{theorem}
\begin{proof}
    \leanok
    \uses{thm:not_exists_int_three_dvd_sub, thm:mem, lmm:lambda_sq}
    % TODO
\end{proof}

\begin{lemma}
    \label{lmm:norm_lambda}
    \lean{IsCyclotomicExtension.Rat.Three.norm_lambda}
    \leanok
    Let $K = \Q(\zeta_3)$ be the third cyclotomic field. \\
    Let $\cc{O}_K = \Z[\zeta_3]$ be the ring of integers of $K$. \\
    Let $\cc{O}^\times_K$ be the group of units of $\cc{O}_K$. \\
    Let $\zeta_3 \in K$ be any primitive third root of unity. \\
    Let $\eta \in \cc{O}_K$ be the element corresponding to $\zeta_3 \in K$. \\
    Let $\lambda \in \cc{O}_K$ be such that $\lambda = \eta -1$. \\\\
    Then the norm of $\lambda$ is $3$.
\end{lemma}
\begin{proof}
    \leanok
    % TODO
\end{proof}

\begin{lemma}
    \label{lmm:norm_lambda_prime}
    \lean{IsCyclotomicExtension.Rat.Three.norm_lambda_prime}
    \leanok
    Let $K = \Q(\zeta_3)$ be the third cyclotomic field. \\
    Let $\cc{O}_K = \Z[\zeta_3]$ be the ring of integers of $K$. \\
    Let $\cc{O}^\times_K$ be the group of units of $\cc{O}_K$. \\
    Let $\zeta_3 \in K$ be any primitive third root of unity. \\
    Let $\eta \in \cc{O}_K$ be the element corresponding to $\zeta_3 \in K$. \\
    Let $\lambda \in \cc{O}_K$ be such that $\lambda = \eta -1$. \\\\
    Then the norm of $\lambda$ is a prime number.
\end{lemma}
\begin{proof}
    \leanok
    \uses{lmm:norm_lambda}
    It directly follows from \Cref{lmm:norm_lambda} since $3$ is a prime number.
\end{proof}

\begin{lemma}
    \label{lmm:lambda_dvd_three}
    \lean{IsCyclotomicExtension.Rat.Three.lambda_dvd_three}
    \leanok
    Let $K = \Q(\zeta_3)$ be the third cyclotomic field. \\
    Let $\cc{O}_K = \Z[\zeta_3]$ be the ring of integers of $K$. \\
    Let $\cc{O}^\times_K$ be the group of units of $\cc{O}_K$. \\
    Let $\zeta_3 \in K$ be any primitive third root of unity. \\
    Let $\eta \in \cc{O}_K$ be the element corresponding to $\zeta_3 \in K$. \\
    Let $\lambda \in \cc{O}_K$ be such that $\lambda = \eta -1$. \\\\
    Then $\lambda \divides 3$.
\end{lemma}
\begin{proof}
    \leanok
    \uses{lmm:norm_lambda}
    By properties of norms and divisibility, if the norm of an element in the ring of integers
    divides a number, then the element itself must divide that number.
    In this case, by \Cref{lmm:norm_lambda} we know that the norm of $\lambda$ is $3$, that divides $3$,
    which implies that $\lambda \divides 3$.
\end{proof}

\begin{theorem}
    \label{thm:zeta_sub_one_prime1}
    \lean{IsPrimitiveRoot.zeta_sub_one_prime'}
    \leanok
    Let $p \in \N$ be prime. \\\\
    If $\zeta_p$ is a primitive $p$-th root of unity, then $\zeta_p - 1$ is prime.
\end{theorem}
\begin{proof}
    \leanok
    This has already been formalised and included in \href{https://pitmonticone.github.io/FLT3/docs/FLT3/Mathlib/NumberTheory/Cyclotomic/Rat.html#IsPrimitiveRoot.zeta_sub_one_prime'}{Mathlib}.
\end{proof}

\begin{lemma}
    \label{lmm:lambda_prime}
    \lean{IsPrimitiveRoot.lambda_prime}
    \leanok
    Let $K = \Q(\zeta_3)$ be the third cyclotomic field. \\
    Let $\cc{O}_K = \Z[\zeta_3]$ be the ring of integers of $K$. \\
    Let $\cc{O}^\times_K$ be the group of units of $\cc{O}_K$. \\
    Let $\zeta_3 \in K$ be any primitive third root of unity. \\
    Let $\eta \in \cc{O}_K$ be the element corresponding to $\zeta_3 \in K$. \\
    Let $\lambda \in \cc{O}_K$ be such that $\lambda = \eta -1$. \\\\
    Then $\lambda$ is prime.
\end{lemma}
\begin{proof}
    \leanok
    \uses{thm:zeta_sub_one_prime1}
    Since $3$ is prime and $\zeta_3$ is a primitive third root of unity, then $\lambda$ is prime
    by \Cref{thm:zeta_sub_one_prime1}.
\end{proof}

\begin{lemma}
    \label{lmm:lambda_ne_zero}
    \lean{IsCyclotomicExtension.Rat.Three.lambda_ne_zero}
    \leanok
    Let $K = \Q(\zeta_3)$ be the third cyclotomic field. \\
    Let $\cc{O}_K = \Z[\zeta_3]$ be the ring of integers of $K$. \\
    Let $\cc{O}^\times_K$ be the group of units of $\cc{O}_K$. \\
    Let $\zeta_3 \in K$ be any primitive third root of unity. \\
    Let $\eta \in \cc{O}_K$ be the element corresponding to $\zeta_3 \in K$. \\
    Let $\lambda \in \cc{O}_K$ be such that $\lambda = \eta -1$. \\\\
    Then $\lambda \neq 0$.
\end{lemma}
\begin{proof}
    \leanok
    \uses{lmm:lambda_prime}
    It directly follows from \Cref{lmm:lambda_prime} since prime numbers are not zero.
\end{proof}

\begin{lemma}
    \label{lmm:lambda_not_unit}
    \lean{IsCyclotomicExtension.Rat.Three.lambda_not_unit}
    \leanok
    Let $K = \Q(\zeta_3)$ be the third cyclotomic field. \\
    Let $\cc{O}_K = \Z[\zeta_3]$ be the ring of integers of $K$. \\
    Let $\cc{O}^\times_K$ be the group of units of $\cc{O}_K$. \\
    Let $\zeta_3 \in K$ be any primitive third root of unity. \\
    Let $\eta \in \cc{O}_K$ be the element corresponding to $\zeta_3 \in K$. \\
    Let $\lambda \in \cc{O}_K$ be such that $\lambda = \eta -1$. \\\\
    Then $\lambda$ is not a unit.
\end{lemma}
\begin{proof}
    \leanok
    \uses{lmm:lambda_prime}
    It directly follows from \Cref{lmm:lambda_prime} since prime numbers are not units.
\end{proof}

\begin{lemma}
    \label{lmm:card_quot}
    \lean{IsCyclotomicExtension.Rat.Three.card_quot}
    \leanok
    Let $K = \Q(\zeta_3)$ be the third cyclotomic field. \\
    Let $\cc{O}_K = \Z[\zeta_3]$ be the ring of integers of $K$. \\
    Let $\cc{O}^\times_K$ be the group of units of $\cc{O}_K$. \\
    Let $\zeta_3 \in K$ be any primitive third root of unity. \\
    Let $\eta \in \cc{O}_K$ be the element corresponding to $\zeta_3 \in K$. \\
    Let $\lambda \in \cc{O}_K$ be such that $\lambda = \eta -1$. \\
    Let $I$ be the ideal generated by $\lambda$. \\\\
    Then $\cc{O}_K / I$ has cardinality $3$.
\end{lemma}
\begin{proof}
    \leanok
    \uses{lmm:norm_lambda}
    It directly follows from \Cref{lmm:norm_lambda} by the fundamental properties of ideals.
\end{proof}

\begin{lemma}
    \label{lmm:two_ne_zero}
    \lean{IsCyclotomicExtension.Rat.Three.two_ne_zero}
    \leanok
    Let $K = \Q(\zeta_3)$ be the third cyclotomic field. \\
    Let $\cc{O}_K = \Z[\zeta_3]$ be the ring of integers of $K$. \\
    Let $\cc{O}^\times_K$ be the group of units of $\cc{O}_K$. \\
    Let $\zeta_3 \in K$ be any primitive third root of unity. \\
    Let $\eta \in \cc{O}_K$ be the element corresponding to $\zeta_3 \in K$. \\
    Let $\lambda \in \cc{O}_K$ be such that $\lambda = \eta -1$. \\
    Let $I$ be the ideal generated by $\lambda$. \\
    Let $2 \in \cc{O}_K ⧸ I$. \\\\
    Then $2 \neq 0$.
\end{lemma}
\begin{proof}
    \leanok
    \uses{lmm:norm_lambda}
    By contradiction we assume that $2 \in I$, then, by definition,
    $\lambda$ would divide $2 \in \cc{O}_K$.
    Recall from \Cref{lmm:norm_lambda} that the norm of $\lambda$ is $3$.
    If $\lambda$ divided $2$, then by properties of divisibility in number fields,
    the norm of $\lambda$ would also divide $2$.
    However $3 \notdivides 2$ showing a contradiction.
    Therefore, $\lambda \notdivides 2$, then $2 \notin I$, which implies
    that $2 \in \mathcal{O}_K / I$ is non-zero.
\end{proof}

\begin{lemma}
    \label{lmm:lambda_not_dvd_two}
    \lean{IsCyclotomicExtension.Rat.Three.lambda_not_dvd_two}
    \leanok
    Let $K = \Q(\zeta_3)$ be the third cyclotomic field. \\
    Let $\cc{O}_K = \Z[\zeta_3]$ be the ring of integers of $K$. \\
    Let $\cc{O}^\times_K$ be the group of units of $\cc{O}_K$. \\
    Let $\zeta_3 \in K$ be any primitive third root of unity. \\
    Let $\eta \in \cc{O}_K$ be the element corresponding to $\zeta_3 \in K$. \\
    Let $\lambda \in \cc{O}_K$ be such that $\lambda = \eta -1$. \\\\
    Then $\lambda \notdivides 2$.
\end{lemma}
\begin{proof}
    \leanok
    \uses{lmm:two_ne_zero}
    By contradiction we assume that $\lambda \divides 2$, that implies that $2 \in I$,
    from which it follows that $2 = 0$ contradicting \Cref{lmm:two_ne_zero}
    that claims that $2 \neq 0$. Therefore, $\lambda \notdivides 2$.
\end{proof}

\begin{lemma}
    \label{lmm:univ_quot}
    \lean{IsCyclotomicExtension.Rat.Three.univ_quot}
    \leanok
    Let $K = \Q(\zeta_3)$ be the third cyclotomic field. \\
    Let $\cc{O}_K = \Z[\zeta_3]$ be the ring of integers of $K$. \\
    Let $\cc{O}^\times_K$ be the group of units of $\cc{O}_K$. \\
    Let $\zeta_3 \in K$ be any primitive third root of unity. \\
    Let $\eta \in \cc{O}_K$ be the element corresponding to $\zeta_3 \in K$. \\
    Let $\lambda \in \cc{O}_K$ be such that $\lambda = \eta -1$. \\
    Let $I$ be the ideal generated by $\lambda$. \\\\
    Then $\cc{O}_K / I = \set{0, 1, -1}$.
\end{lemma}
\begin{proof}
    \leanok
    \uses{lmm:card_quot, lmm:two_ne_zero}
    It is obvious that $\cc{O}_K / I \supseteq \set{0, 1, -1}$. Moreover, by \Cref{lmm:card_quot},
    the cardinality of $\cc{O}_K / I$ is $3$, so it suffices to prove that $1,-1$ and $0$ are distinct.\\
    We proceed by contradiction analysing each case:
    \begin{itemize}
        \item Case $1 = -1$. By algebraic properties, $1 = -1$ implies that $2 = 0$,
              which contradicts \Cref{lmm:two_ne_zero} forcing us to conclude that $1 \neq -1$.
        \item Case $1 = 0$. Trivial contradiction.
        \item Case $-1 = 0$. It implies that $1 = 0$, which is a contradiction.
    \end{itemize}
\end{proof}

\begin{lemma}
    \label{lmm:dvd_or_dvd_sub_one_or_dvd_add_one}
    \lean{IsCyclotomicExtension.Rat.Three.dvd_or_dvd_sub_one_or_dvd_add_one}
    \leanok
    Let $K = \Q(\zeta_3)$ be the third cyclotomic field. \\
    Let $\cc{O}_K = \Z[\zeta_3]$ be the ring of integers of $K$. \\
    Let $\cc{O}^\times_K$ be the group of units of $\cc{O}_K$. \\
    Let $\zeta_3 \in K$ be any primitive third root of unity. \\
    Let $\eta \in \cc{O}_K$ be the element corresponding to $\zeta_3 \in K$. \\
    Let $\lambda \in \cc{O}_K$ be such that $\lambda = \eta -1$. \\
    Let $x \in \cc{O}_K$. \\\\
    Then $(\lambda \divides x) \lor (\lambda \divides x-1) \lor (\lambda \divides x+1)$.
\end{lemma}
\begin{proof}
    \leanok
    \uses{lmm:univ_quot}
    Let $I$ be the ideal generated by $\lambda$. Let $\pi : \cc{O}_K \to \cc{O}_K / I$.\\
    By \Cref{lmm:univ_quot}, we have that $\pi(x) \in \cc{O}_K / I = \set{0, 1, -1}$.\\
    We proceed by analysing each case:
    \begin{itemize}
        \item Case $\pi(x) = 0$. By properties of ideals, $\lambda \divides x$.
        \item Case $\pi(x) = 1$. Then $0=\pi(x)-1=\pi(x-1)$, which, by properties of ideals,
        implies that $\lambda \divides x-1$.
        \item Case $\pi(x) = -1$. Then $0=\pi(x)+1=\pi(x+1)$, which, by properties of ideals,
        implies that $\lambda \divides x+1$.
    \end{itemize}
\end{proof}

\begin{lemma}
    \label{lmm:toInteger_cube_eq_one}
    \lean{IsPrimitiveRoot.toInteger_cube_eq_one}
    \leanok
    Let $K = \Q(\zeta_3)$ be the third cyclotomic field. \\
    Let $\cc{O}_K = \Z[\zeta_3]$ be the ring of integers of $K$. \\
    Let $\cc{O}^\times_K$ be the group of units of $\cc{O}_K$. \\
    Let $\zeta_3 \in K$ be any primitive third root of unity. \\
    Let $\eta \in \cc{O}_K$ be the element corresponding to $\zeta_3 \in K$. \\\\
    Then $\eta^3 = 1$.
\end{lemma}
\begin{proof}
    \leanok
    Since $\zeta_3 \in K$ is a primitive third root of unity, then $\zeta_3^3 = 1$.
    Given that $\eta \in \cc{O}_K$ is the element corresponding to $\zeta_3 \in K$, then
    $\eta^3 = 1$ by the extension of the field properties into the ring of integers.
\end{proof}

\begin{lemma}
    \label{lmm:eta_isUnit}
    \lean{IsPrimitiveRoot.eta_isUnit}
    \leanok
    Let $K = \Q(\zeta_3)$ be the third cyclotomic field. \\
    Let $\cc{O}_K = \Z[\zeta_3]$ be the ring of integers of $K$. \\
    Let $\cc{O}^\times_K$ be the group of units of $\cc{O}_K$. \\
    Let $\zeta_3 \in K$ be any primitive third root of unity. \\
    Let $\eta \in \cc{O}_K$ be the element corresponding to $\zeta_3 \in K$. \\\\
    Then $\eta$ is a unit.
\end{lemma}
\begin{proof}
    \leanok
    \uses{lmm:toInteger_cube_eq_one}
    It directly follows from \Cref{lmm:toInteger_cube_eq_one}.
\end{proof}

\begin{lemma}
    \label{lmm:toInteger_eval_cyclo}
    \lean{IsPrimitiveRoot.toInteger_eval_cyclo}
    \leanok
    Let $K = \Q(\zeta_3)$ be the third cyclotomic field. \\
    Let $\cc{O}_K = \Z[\zeta_3]$ be the ring of integers of $K$. \\
    Let $\cc{O}^\times_K$ be the group of units of $\cc{O}_K$. \\
    Let $\zeta_3 \in K$ be any primitive third root of unity. \\
    Let $\eta \in \cc{O}_K$ be the element corresponding to $\zeta_3 \in K$. \\\\
    Then $\eta^2 + \eta + 1 = 0$.
\end{lemma}
\begin{proof}
    \leanok
    Since $\eta$ corresponds to a root of the equation $x^2 + x + 1 = 0$,
    then $\eta^2 + \eta + 1 = 0$.
\end{proof}

\begin{lemma}
    \label{lmm:cube_sub_one}
    \lean{IsCyclotomicExtension.Rat.Three.cube_sub_one}
    \leanok
    Let $K = \Q(\zeta_3)$ be the third cyclotomic field. \\
    Let $\cc{O}_K = \Z[\zeta_3]$ be the ring of integers of $K$. \\
    Let $\cc{O}^\times_K$ be the group of units of $\cc{O}_K$. \\
    Let $\zeta_3 \in K$ be any primitive third root of unity. \\
    Let $\eta \in \cc{O}_K$ be the element corresponding to $\zeta_3 \in K$. \\
    Let $x \in \cc{O}_K$. \\\\
    Then $x^3 - 1 = (x - 1)(x - \eta)(x - \eta^ 2)$.
\end{lemma}
\begin{proof}
    \leanok
    \uses{lmm:toInteger_cube_eq_one, lmm:toInteger_eval_cyclo}
    Applying \Cref{lmm:toInteger_cube_eq_one} and \Cref{lmm:toInteger_eval_cyclo}, we have that
    \begin{align*}
        (x - 1)(x - \eta)(x - \eta^ 2)
        &= x^3 - x^2 (\eta^2 + \eta + 1) + x (\eta^2 + \eta + \eta^3) - \eta^3 \\
        &= x^3 - x^2 (\eta^2 + \eta + 1) + x (\eta^2 + \eta + 1) - 1 \\
        &= x^3 - 1.
    \end{align*}
\end{proof}

\begin{lemma}
    \label{lmm:lambda_dvd_mul_sub_one_mul_sub_eta_add_one}
    \lean{IsCyclotomicExtension.Rat.Three.lambda_dvd_mul_sub_one_mul_sub_eta_add_one}
    \leanok
    Let $K = \Q(\zeta_3)$ be the third cyclotomic field. \\
    Let $\cc{O}_K = \Z[\zeta_3]$ be the ring of integers of $K$. \\
    Let $\cc{O}^\times_K$ be the group of units of $\cc{O}_K$. \\
    Let $\zeta_3 \in K$ be any primitive third root of unity. \\
    Let $\eta \in \cc{O}_K$ be the element corresponding to $\zeta_3 \in K$. \\
    Let $\lambda \in \cc{O}_K$ be such that $\lambda = \eta -1$. \\
    Let $x \in \cc{O}_K$. \\\\
    Then $\lambda \divides x(x - 1)(x - (\eta + 1))$.
\end{lemma}
\begin{proof}
    \leanok
    \uses{lmm:dvd_or_dvd_sub_one_or_dvd_add_one, lmm:lambda_dvd_three}
    By \Cref{lmm:dvd_or_dvd_sub_one_or_dvd_add_one}, we have that
    $$(\lambda \divides x) \lor (\lambda \divides x-1) \lor (\lambda \divides x+1).$$
    We proceed by analysing each case:
    \begin{itemize}
        \item Case $\lambda \divides x$. By properties of divisibility
              $\lambda \divides x(x - 1)(x - (\eta + 1))$.
        \item Case $\lambda \divides x-1$. By properties of divisibility
              $\lambda \divides x(x - 1)(x - (\eta + 1))$.
        \item Case $\lambda \divides x+1$.\\
              By properties of divisibility it suffices to prove that
              $$\lambda \divides x - (\eta + 1) = x + 1 - (\eta - 1 + 3).$$
              By definition of $\lambda$, we have that
              $$x + 1 - (\eta - 1 + 3) = x + 1 - (\lambda + 3).$$
              By properties of divisibility and \Cref{lmm:lambda_dvd_three},
              $\lambda \divides \lambda + 3$.\\
              Therefore, by properties of divisibility, we can conclude that
              $$\lambda \divides x(x - 1)(x - (\eta + 1)).$$
    \end{itemize}
\end{proof}

\begin{lemma}
    \label{lmm:lambda_pow_four_dvd_cube_sub_one_of_dvd_sub_one}
    \lean{IsCyclotomicExtension.Rat.Three.lambda_pow_four_dvd_cube_sub_one_of_dvd_sub_one}
    \leanok
    Let $K = \Q(\zeta_3)$ be the third cyclotomic field. \\
    Let $\cc{O}_K = \Z[\zeta_3]$ be the ring of integers of $K$. \\
    Let $\cc{O}^\times_K$ be the group of units of $\cc{O}_K$. \\
    Let $\zeta_3 \in K$ be any primitive third root of unity. \\
    Let $\eta \in \cc{O}_K$ be the element corresponding to $\zeta_3 \in K$. \\
    Let $\lambda \in \cc{O}_K$ be such that $\lambda = \eta -1$. \\
    Let $x \in \cc{O}_K$. \\\\
    If $\lambda\divides x - 1$, then $\lambda^ 4 \divides x^3 - 1$.
\end{lemma}
\begin{proof}
    \leanok
    \uses{lmm:cube_sub_one, lmm:lambda_dvd_mul_sub_one_mul_sub_eta_add_one}
    Let $y\in \cc{O}_K$. \\
    Let $\lambda \divides x - 1$, which implies that $x - 1 = \lambda y$. \\
    By ring properties and \Cref{lmm:cube_sub_one}, we have that
    $$x^3 - 1 = \lambda^3 (y (y - 1) (y - (\eta + 1))).$$
    By divisibility properties and \Cref{lmm:lambda_dvd_mul_sub_one_mul_sub_eta_add_one},
    we can conclude that $$\lambda^ 4 \divides x^3 - 1.$$
\end{proof}

\begin{lemma}
    \label{lmm:lambda_pow_four_dvd_cube_add_one_of_dvd_add_one}
    \lean{IsCyclotomicExtension.Rat.Three.lambda_pow_four_dvd_cube_add_one_of_dvd_add_one}
    \leanok
    Let $K = \Q(\zeta_3)$ be the third cyclotomic field. \\
    Let $\cc{O}_K = \Z[\zeta_3]$ be the ring of integers of $K$. \\
    Let $\cc{O}^\times_K$ be the group of units of $\cc{O}_K$. \\
    Let $\zeta_3 \in K$ be any primitive third root of unity. \\
    Let $\eta \in \cc{O}_K$ be the element corresponding to $\zeta_3 \in K$. \\
    Let $\lambda \in \cc{O}_K$ be such that $\lambda = \eta -1$. \\
    Let $x \in \cc{O}_K$. \\\\
    If $\lambda \divides x + 1$, then $\lambda^4 \divides x ^ 3 + 1$.
\end{lemma}
\begin{proof}
    \leanok
    \uses{lmm:lambda_pow_four_dvd_cube_sub_one_of_dvd_sub_one}
    Let $y\in \cc{O}_K$. \\
    Let $\lambda \divides x + 1 = - x - 1$, which implies that $- x - 1 = \lambda y$. \\
    By \Cref{lmm:lambda_dvd_mul_sub_one_mul_sub_eta_add_one}, we have that
    $$\lambda^ 4 \divides (-x)^3 - 1.$$
    By ring properties we can conclude that $$\lambda^ 4 \divides x^3 + 1.$$
\end{proof}

\begin{lemma}
    \label{lmm:lambda_pow_four_dvd_cube_sub_one_or_add_one_of_lambda_not_dvd}
    \lean{IsCyclotomicExtension.Rat.Three.lambda_pow_four_dvd_cube_sub_one_or_add_one_of_lambda_not_dvd}
    \leanok
    Let $K = \Q(\zeta_3)$ be the third cyclotomic field. \\
    Let $\cc{O}_K = \Z[\zeta_3]$ be the ring of integers of $K$. \\
    Let $\cc{O}^\times_K$ be the group of units of $\cc{O}_K$. \\
    Let $\zeta_3 \in K$ be any primitive third root of unity. \\
    Let $\eta \in \cc{O}_K$ be the element corresponding to $\zeta_3 \in K$. \\
    Let $\lambda \in \cc{O}_K$ be such that $\lambda = \eta -1$. \\
    Let $x \in \cc{O}_K$. \\\\
    If $\lambda \notdivides x$, then $(\lambda^4 \divides x^3 - 1)
    \lor (\lambda^4 \divides x^3 + 1)$.
\end{lemma}
\begin{proof}
    \leanok
    \uses{lmm:dvd_or_dvd_sub_one_or_dvd_add_one,
    lmm:lambda_pow_four_dvd_cube_sub_one_of_dvd_sub_one,
    lmm:lambda_pow_four_dvd_cube_add_one_of_dvd_add_one}
    By \Cref{lmm:dvd_or_dvd_sub_one_or_dvd_add_one}, we have that
    $$(\lambda \divides x) \lor (\lambda \divides x-1) \lor (\lambda \divides x+1).$$
    We proceed by analysing each case:
    \begin{itemize}
        \item Case $\lambda \divides x$. From trivially contradictory hypotheses we can conclude that
        $$(\lambda^4 \divides x^3 - 1) \lor (\lambda^4 \divides x^3 + 1).$$
        \item Case $\lambda \divides x-1$. By \Cref{lmm:lambda_pow_four_dvd_cube_sub_one_of_dvd_sub_one},
        we have that $\lambda^ 4 \divides x^3 - 1$, which implies that
        $$(\lambda^4 \divides x^3 - 1) \lor (\lambda^4 \divides x^3 + 1).$$
        \item Case $\lambda \divides x+1$. By \Cref{lmm:lambda_pow_four_dvd_cube_add_one_of_dvd_add_one},
        we have that $\lambda^ 4 \divides x^3 + 1$, which implies that
        $$(\lambda^4 \divides x^3 - 1) \lor (\lambda^4 \divides x^3 + 1).$$
    \end{itemize}
\end{proof}