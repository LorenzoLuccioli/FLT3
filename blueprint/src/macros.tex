%%%%%%%%%% THEOREM ENVIRONMENTS %%%%%%%%%%
%\theoremstyle{•}
% • plain: This is the default style. The header (e.g., "Theorem 1") is bolded and the body text is in italic. Both are on their own line.
% • definition: The header is bolded, but the body text is not italicized. Both are on their own line. This style is typically used for definitions, examples, etc.
% • remark: The header is italicized, the body text is not italicized, and both are run-in; that is, they appear on the same line.

% "Definition" style for definitions and axioms
\theoremstyle{definition}
\newtheorem{definition}{Definition}
\newtheorem{axiom}{Axiom}

% "Plain" style for theorems, lemmas, corollaries and propositions
\theoremstyle{plain}
\newtheorem{theorem}{Theorem}[section]
\newtheorem{lemma}[theorem]{Lemma}
\newtheorem{corollary}[theorem]{Corollary}
\newtheorem{proposition}[theorem]{Proposition}

% "Remark" style for examples, exercises, remarks, observations, solutions and notations
\theoremstyle{remark}
\newtheorem{example}[theorem]{Example}
\newtheorem{non-example}[theorem]{Non Example}
\newtheorem{exercise}[theorem]{Exercise}
\newtheorem*{remark}{Remark}
\newtheorem*{observation}{Observation}
\newtheorem*{solution}{Solution}
\newtheorem*{notation}{Notation}

%%%%%%%%%% BOXED ENVIRONMENTS %%%%%%%%%%

%\newtcbtheorem[number within=section, counter*=definition]{bthing}{Thing}{%
%    colframe=white!10!black,
%    colback=white!97!black,
%    coltitle=white,
%    fonttitle=\bfseries,
%}{bthng}

% Notation Box
\newtcbtheorem{bnotation}{Notation}{%
    colframe=white!10!black,
    colback=white!97!black,
    coltitle=white,
    fonttitle=\bfseries,
}{bcpt}

% Concept Box
\newtcbtheorem{bconcept}{Concept}{%
    colframe=white!10!black,
    colback=white!97!black,
    coltitle=white,
    fonttitle=\bfseries,
}{bcpt}

% Axiom Box
\newtcbtheorem{baxiom}{Axiom}{%
    colframe=white!10!black,
    colback=white!97!black,
    coltitle=white,
    fonttitle=\bfseries,
}{bax}

% Definition Box
%%%\newtcbtheorem[number within=section, use counter=definition]{bdefinition}{Definition}{%
\newtcbtheorem[use counter=definition]{bdefinition}{Definition}{%
    colframe=white!10!black,
    colback=white!97!black,
    coltitle=white,
    fonttitle=\bfseries,
}{bdef}


% Theorem Box
\newtcbtheorem{btheorem}{Theorem}{%
    colframe=white!36!black,
    colback=white!97!black,
    coltitle=white,
    fonttitle=\bfseries,
}{bthm}

% Lemma Box
\newtcbtheorem{blemma}{Lemma}{%
    colframe=white!36!black,
    colback=white!97!black,
    coltitle=white,
    fonttitle=\bfseries,
}{blmm}

% Proposition Box
\newtcbtheorem{bproposition}{Proposition}{%
    colframe=white!36!black,
    colback=white!97!black,
    coltitle=white,
    fonttitle=\bfseries,
}{bprop}

%%%%%%%%%%%% NOTATIONS %%%%%%%%%%%%

% Links
\newcommand{\defref}[2]{\hyperref[def:#1]{\textbf{#2}}}
\newcommand{\axref}[2]{\hyperref[ax:#1]{\textbf{#2}}}
\newcommand{\thmref}[2]{\hyperref[thm:#1]{\textbf{#2}}}
\newcommand{\lmmref}[2]{\hyperref[lmm:#1]{\textbf{#2}}}
\newcommand{\bdefref}[2]{\hyperref[bdef:#1]{\textbf{#2}}}
\newcommand{\baxref}[2]{\hyperref[bax:#1]{\textbf{#2}}}
\newcommand{\bthmref}[2]{\hyperref[bthm:#1]{\textbf{#2}}}
\newcommand{\blmmref}[2]{\hyperref[blmm:#1]{\textbf{#2}}}

% Basics
\newcommand{\bb}[1]{\mathbb{#1}} % Blackboard bold font
\newcommand{\cc}[1]{\mathcal{#1}} % Calligraphic font
\newcommand{\dd}[1]{\mathrm{d}#1} % Differential operator

% Set
\newcommand{\set}[1]{\ensuremath{\left\{#1\right\}}}
\newcommand{\setb}[2]{\ensuremath{\left\{#1\mid#2\right\}}}
%\newcommand{\set}[2]{\ensuremath{\left\{#1:#2\right\}}}

% Functions
\newcommand{\fun}[3]{\ensuremath{#1:#2\to#3}}
\newcommand{\id}{\ensuremath{\mathrm{id}}}

% Number sets
\newcommand{\N}{\ensuremath{\mathbb{N}}} % Natural
\newcommand{\Z}{\ensuremath{\mathbb{Z}}} % Integer
\newcommand{\Q}{\ensuremath{\mathbb{Q}}} % Rational
\newcommand{\R}{\ensuremath{\mathbb{R}}} % Real
\newcommand{\C}{\ensuremath{\mathbb{C}}} % Complex
\newcommand{\K}{\ensuremath{\mathbb{K}}} % Generic Field
\newcommand{\F}{\ensuremath{\mathbb{F}}} % Generic Field

% Mathematical logic
\newcommand{\imp}{\rightarrow} % Implication
\newcommand{\biimp}{\leftrightarrow} % Biconditional
\newcommand{\all}{\forall} % Universal quantifier
\newcommand{\ex}{\exists} % Existential quantifier
\newcommand{\sat}{\models} % Satisfies
\newcommand{\ent}{\vdash} % Entails
\newcommand{\nent}{\nvdash} % Does not entail
%\newcommand{\proves}{\vdash} % Proves
\newcommand{\nproves}{\nvdash} % Does not prove
\newcommand{\true}{\top} % Tautology
\newcommand{\false}{\bot} % Contradiction

% Others
\newcommand{\abs}[1]{\left|#1\right|} % Absolute value
\newcommand{\avg}[1]{\left\langle#1\right\rangle} % Average value
\newcommand{\norm}[1]{\left\|#1\right\|} % Norm
\newcommand{\bra}[1]{\left\langle#1\right|} % Bra notation
\newcommand{\ket}[1]{\left|#1\right\rangle} % Ket notation
\newcommand{\braket}[2]{\left\langle#1\mid|#2\right\rangle} % Bra-ket notation
\newcommand{\expect}[1]{\left\langle#1\right\rangle} % Expectation value
\newcommand{\op}[1]{\hat{#1}} % Operator notation
\newcommand{\comm}[2]{\left[#1,#2\right]} % Commutator
\newcommand{\acomm}[2]{\left\{#1,#2\right\}} % Anticommutator
\newcommand{\tr}[1]{\mathrm{Tr}\left[#1\right]} % Trace
\newcommand{\diag}[1]{\mathrm{diag}\left(#1\right)} % Diagonal matrix
\newcommand{\rank}[1]{\mathrm{rank}\left(#1\right)} % Rank of a matrix
\newcommand{\erf}[1]{\mathrm{erf}\left(#1\right)} % Error function
\newcommand{\erfc}[1]{\mathrm{erfc}\left(#1\right)} % Complementary error function
\newcommand{\sinc}[1]{\mathrm{sinc}\left(#1\right)} % Sinc function
\newcommand{\sign}[1]{\mathrm{sign}\left(#1\right)} % Sign function
\newcommand{\floor}[1]{\left\lfloor#1\right\rfloor} % Floor function
\newcommand{\ceil}[1]{\left\lceil#1\right\rceil} % Ceiling function
\newcommand{\round}[1]{\left\lfloor#1\right\rceil} % Rounding function
\newcommand{\argmin}[1]{\mathrm{argmin}\left\{#1\right\}} % Argmin
\newcommand{\argmax}[1]{\mathrm{argmax}\left\{#1\right\}} % Argmax