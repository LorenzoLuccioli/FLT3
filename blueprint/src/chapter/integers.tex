\chapter{The integer case}

\begin{theorem}
\label{th-ap-int}
\uses{bohr-reg-def}
There is a constant $c>0$ such that the following holds. Let $\epsilon>0$ and $B,B'\subseteq G$ be regular Bohr sets of rank $d$. Suppose that $A_1\subseteq B$ with density $\alpha_1$ and $A_2$ is such that there exists $x$ with $A_2\subseteq B'-x$ with density $\alpha_2$. Let $S$ be any set with $\abs{S}\leq 2\abs{B}$. There is a regular Bohr set $B''\subseteq B'$ of rank at most
\[d+O_\epsilon(\lo{\alpha_1}^3\lo{\alpha_2})\]
and size
\[\abs{B''}\geq \exp(-O_\epsilon(d\lo{\alpha_1\alpha_2/d}+\lo{\alpha_1}^3\lo{\alpha_2}\lo{\alpha_1\alpha_2/d}))\abs{B'}\]
such that
\[\abs{\langle \mu_{B'}\ast \mu_{A_1}\circ \mu_{A_2},\ind{S}\rangle-\langle \mu_{A_1}\circ \mu_{A_2},\ind{S}\rangle}\leq \epsilon.\]
\end{theorem}
\begin{proof}
\uses{linfty_ap, chang, bohr-size}
To do.
\end{proof}

\begin{proposition}
\label{Lp-orth}
\uses{bohr-reg-def}
There is a constant $c>0$ such that the following holds. Let $\epsilon >0$ and $p \geq 2$ be an integer. Let $B \subseteq G$ be a regular Bohr set and $A\subseteq B$ with relative density $\alpha$. Let $\nu : G \to \bbr_{\geq 0}$ be supported on $B_\rho$, where $\rho \leq c\epsilon\alpha/\rk(B)$, such that $\norm{\nu}_1=1$ and $\widehat{\nu}\geq 0$. If
    \[ \norm{(\mu_A-\mu_B) \circ (\mu_{A}-\mu_B)}_{p(\nu)} \geq \epsilon\, \mu(B)^{-1}, \]
    then there exists $p'\ll_\epsilon p$ such that
    \[ \norm{ \mu_{A}\circ \mu_{A}}_{p'(\nu)} \geq \left(1+\tfrac{1}{4}\epsilon\right) \mu(B)^{-1}. \]
\end{proposition}
\begin{proof}
\uses{reg-conv, unbalancing}
To do.
\end{proof}


\begin{proposition}
\label{pos-def-measures}
\uses{bohr-reg-def}
There is a constant $c>0$ such that the following holds. Let $p \geq 2$ be an even integer. Let $f : G \to \bbr$, let $B \subseteq G$ and $B', B'' \subseteq B_{c/\rk(B)}$ all be regular Bohr sets. Then
\[ \norm{ f\circ f }_{p(\mu_{B'}\circ \mu_{B'}\ast \mu_{B''}\circ \mu_{B''})} \geq \tfrac{1}{2} \norm{f*f}_{p(\mu_B)}. \]
\end{proposition}
\begin{proof}
\uses{bohr-majorise}
To do,
\end{proof}

\begin{proposition}
\label{hoelder-lifting}
\uses{bohr-reg-def}
There is a constant $c>0$ such that the following holds. Let $\epsilon >0$. Let $B \subseteq G$ be a regular Bohr set and $A\subseteq B$ with relative density $\alpha$, and let $B' \subseteq B_{c\epsilon\alpha/\rk(B)}$ be a regular Bohr set and $C\subseteq B'$ with relative density $\gamma$. Either
\begin{enumerate}
\item $\abs{ \langle \mu_A*\mu_A, \mu_{C} \rangle - \mu(B)^{-1} } \leq \epsilon \mu(B)^{-1}$ or
\item there is some $p \ll\lo{\gamma}$ such that $\norm{ (\mu_A-\mu_B)*(\mu_A-\mu_B)}_{p(\mu_{B'})} \geq \tfrac{1}{2}\epsilon \mu(B)^{-1}$.
\end{enumerate}
\end{proposition}
\begin{proof}
%\uses{reg-conv}
To do.
\end{proof}

\begin{proposition}
\label{prop-it}
\uses{bohr-reg-def}
There is a constant $c>0$ such that the following holds. Let $\epsilon>0$ and $p,k\geq 1$ be integers such that $(k,\abs{G})=1$. Let $B,B',B''\subseteq G$ be regular Bohr sets of rank $d$ such that $B''\subseteq B'_{c/d}$ and $A\subseteq B$ with relative density $\alpha$. If
    \[ \norm{ \mu_{A}\circ \mu_{A}}_{p(\mu_{k\cdot B'}\circ\mu_{k\cdot B'}\ast \mu_{k\cdot B''}\circ \mu_{k\cdot B''})} \geq \left(1+\epsilon\right) \mu(B)^{-1},\]
    then there is a regular Bohr set $B'''\subseteq B''$ of rank at most
    \[\rk(B''')\leq d+O_{\epsilon}(\lo{\alpha}^4p^4)\]
    and size
    \[\abs{B'''}\geq \exp(-O_{\epsilon}(dp\lo{\alpha/d}+\lo{\alpha}^5p^5))\abs{B''}\]
    such that
    \[ \norm{ \mu_{B'''}*\mu_A }_\infty \geq (1+c\epsilon)\mu(B)^{-1}. \]
\end{proposition}
\begin{proof}
\uses{sift, th-ap-int}
To do.
\end{proof}

\begin{theorem}
\label{flat-on-bohr}
\uses{bohr-reg-def}
There is a constant $c>0$ such that the following holds. Let ${\epsilon,\delta\in (0,1)}$ and $p,k\geq 1$ be integers such that $(k,\abs{G})=1$. For any $A\subseteq G$ with density $\alpha$ there is a regular Bohr set $B$ with
\[ d=\rk(B) =O_{\epsilon}\left(\lo{\alpha}^5p^4\right) \quad\text{and}\quad \abs{B}\geq \exp\left(-O_{\epsilon,\delta}(\lo{\alpha}^6p^5\lo{\alpha/p})\right)\abs{G} \]
and some $A'\subseteq (A-x)\cap B$ for some $x \in G$ such that
\begin{enumerate}
\item $\abs{A'}\geq (1-\epsilon)\alpha\abs{B}$,
\item $\abs{A'\cap B'}\geq (1-\epsilon)\alpha\abs{B'}$, where $B'=B_{\rho}$ is a regular Bohr set with ${\rho\in (\tfrac{1}{2},1)\cdot c\delta\alpha/d}$, and
\item
\[\norm{\mu_{A'}\circ \mu_{A'}}_{p(\mu_{k\cdot B''}\circ \mu_{k\cdot B''}\ast \mu_{k\cdot B'''}\circ \mu_{k\cdot B'''})} <(1+ \epsilon)\mu(B)^{-1},\]
for any regular Bohr sets $B'' = B'_{\rho'}$ and $B'''=B''_{\rho''}$ satisfying ${\rho',\rho''\in(\frac{1}{2},1)\cdot c\delta\alpha/d}$.
\end{enumerate}
\end{theorem}
\begin{proof}
\uses{prop-it, bourgain-trick, bohr-size}
To do.
\end{proof}

\begin{theorem}
\label{th-int-gen}
\uses{bohr-reg-def}
There is a constant $c>0$ such that the following holds. Let $\delta,\epsilon\in (0,1)$, let $p \geq 1$ and let $k$ be a positive integer such that $(k,\abs{G})=1$. There is a constant $C=C(\epsilon,\delta,k)>0$ such that the following holds.

For any finite abelian group $G$ and any subset $A\subseteq G$ with $\abs{A}=\alpha \abs{G}$ there exists a regular Bohr set $B$ with
\[\rk(B)\leq Cp^4\log(2/\alpha)^5\]
and
\[\abs{B}\geq \exp\left(-Cp^5\log(2p/\alpha)\log(2/\alpha)^6\right)\abs{G}\]
and $A' \subseteq (A-x)\cap B$ for some $x\in G$ such that
\begin{enumerate}
\item $\abs{A'}\geq (1-\epsilon)\alpha \abs{B}$,
\item $\abs{A'\cap B'}\geq (1-\epsilon)\alpha\abs{B'}$, where $B'=B_{\rho}$ is a regular Bohr set with $\rho\in (\tfrac{1}{2},1)\cdot c\delta\alpha/dk$, and
\item
\[\norm{(\mu_{A'}-\mu_B)\ast (\mu_{A'}- \mu_B)}_{p(\mu_{k\cdot B'})} \leq \epsilon\frac{\abs{G}}{\abs{B}}.\]
\end{enumerate}
\end{theorem}
\begin{proof}
\uses{Lp-orth, flat-on-bohr, pos-def-measures}
To do.
\end{proof}

\begin{theorem}\label{main-int-count}
If $A\subseteq \{1,\ldots,N\}$ has size $\abs{A}=\alpha N$, then $A$ contains at least
\[\exp(-O(\lo{\alpha}^{12}))N^2\]
many three-term arithmetic progressions.
\end{theorem}
\begin{proof}
\uses{th-int-gen, hoelder-lifting}
To do.
\end{proof}

\begin{theorem}\label{3aps-in-ints}
If $A\subseteq \{1,\ldots,N\}$ contains no non-trivial three-term arithmetic progressions then
\[\lvert A\rvert \leq \frac{N}{\exp(-c(\log N)^{1/12})}\]
for some constant $c>0$.
\end{theorem}
\begin{proof}
\uses{main-int-count}
To do.
\end{proof}
